% File   : LaTeX template (article)
% Author : José Molina Company <cyma.deb@protonmail.ch>
% Source : <https://github.com/cyma/LaTeX>
% License : MIT Copyright (c) 2020-2021 José Molina Company

% PACKAGES AND CONFIGURATIONS

\documentclass[a4paper,twoside,11pt]{article}
\usepackage[utf8]{inputenc} %inputting international characters
\usepackage[T1]{fontenc} % output font encoding
\usepackage[spanish]{babel} %change to your language
\usepackage[svgnames]{xcolor} % color package
\usepackage{fancyhdr} %  page margings and sizes, headers and footers,and the proper placement of figures and tables
\usepackage[hyperfootnotes=false]{hyperref} % create hyperlinks
\usepackage{extramarks} % combine with fancyhdr to give extra customization
\usepackage{geometry} %  customize page layout, implementing auto-centering and auto-balancing mechanisms
\usepackage{graphicx} % key-value interface for optional arguments to the \includegraphics command.
%\graphicspath{{examples/documentation/images/}{images/}} % Paths in which to look for images

\usepackage{setspace} % change spacing within the document
\usepackage{framed} % framed or shaded regions that can break across pages
\usepackage{enumitem} % control layout of itemize, enumerate, description
\usepackage{caption} % customising captions in floating environments
\usepackage{subcaption} % support for sub-captions
%\usepackage{schemata} % print topical diagrams
\usepackage[all]{xy} % A package for typesetting a variety of graphs and diagrams with TeX

\usepackage[official]{eurosym} % euro symbol
\usepackage{amsmath} % mathematical typesetting
\usepackage{amssymb} % mathematical symbols
\usepackage{mathrsfs} % \mathscr script font that is more elaborate than the ``calligraphic'' font obtained by \mathcal
\usepackage{extarrows} % extra Arrows beyond those provided in amsmath
\usepackage{siunitx} % Provides the \SI{}{} and \si{} command for typesetting SI units
\usepackage{relsize} % allow \mathlarger
\usepackage{mathtools} % enhance the appearance of documents containing a lot of mathematics
\usepackage{cancel} % a package to draw diagonal lines (“cancelling” a term) and arrows with limits (cancelling a term “to a value”) through parts of maths formulae.
%\usepackage{svrsymbols} % A font with symbols for use in physics texts
%\usepackage{mhchem} % typeset chemical formulae/equations and Risk and Safety phrases
%\usepackage{bohr} % simple atom representation according to the Bohr model
%\usepackage{elements} % the package provides means for retrieving properties of chemical elements.
%\usepackage{marvosym} % symbols for structural engineering; symbols for steel cross-sections; astronomy signs (sun, moon, planets); the 12 signs of the zodiac; scissor symbols; CE sign and others.
%\usepackage{halloweenmath} % Scary and creepy math symbols

%Verbatim boxed text
\usepackage{fancyvrb,fancybox,calc}
\usepackage[svgnames]{xcolor}
\newenvironment{verbcode}{\VerbatimEnvironment%
  \noindent
  \begin{Sbox}
  \begin{minipage}{\linewidth-2\fboxsep-2\fboxrule-4pt}
  \begin{Verbatim}
}{%
  \end{Verbatim}
  \end{minipage}
  \end{Sbox}
  \fcolorbox{black}{LightGray}{\TheSbox}
}

%Figures
%\begin{figure}[htbp]
%    \begin{center}
%         \includegraphics[scale=]{.png}
%     \end{center}
% \caption{Caption}
% \end{figure}

%Tables
% \begin{table}{htbp}
% \centering
%     \begin{tabular}[ccc]
%
%
%     \end{tabular}
% \caption{}
% \end{table}

%Bibtex
\usepackage{comment}
\usepackage[backend=bibtex,style=alphabetic,autocite=footnote,natbib=true]{biblatex}
\usepackage[autostyle=true]{csquotes} % Required to generate language-dependent quotes in the bibliography

\addbibresource{sample.bib}


\geometry{lmargin=2.5cm,rmargin=2.5cm,tmargin=3cm,bmargin=2.5cm}

\setlength{\parindent}{0pt} % remove indent

%\renewcommand{\labelenumi}{\alph{enumi}.} % Make numbering in the enumerate environment by letter rather than number (e.g. section 6)

%\sloppy % rules to fix some line breaking
%\fuzzy % default LaTeX rules
%\date{}


\begin{document}


% TITLE


\begin{titlepage}
 \newcommand{\HRule}{\rule{\linewidth}{0.5mm}}

	\center

    % Headings

    \textsc{\LARGE \bf\textcolor{red}{Church of Emacs}}\\[1.5cm] % university/college

    \textsc{\Large \textcolor{red}{Vi IMproved}}\\[0.5cm] % Course name

	\textsc{\large A Murloc stole my Hook}\\[0.5cm] % Course title

    % Title

    \HRule\\[0.4cm]

    {\huge\bfseries $\mathwitch$ Number VI-VI-VI $\reversemathwitch$}\\[0.4cm] % Title of your document

	\HRule\\[1.5cm]

    %	Author

    \begin{minipage}{0.4\textwidth}
		\begin{flushleft}
			\large
            \textit{\textcolor{red}{Autor}}\\
			cyma % Your name
		\end{flushleft}
	\end{minipage}
	~
	\begin{minipage}{0.4\textwidth}
		\begin{flushright}
			\large
            \textit{\textcolor{red}{Supervisor}}\\
			Flying Spaghetti Monster% Supervisor's name
		\end{flushright}
	\end{minipage}

	% If you don't want a supervisor, uncomment the two lines below and comment the code above
	%{\large\textit{Author}}\\
	%The \textsc{Coconut} % Oil

    % Logo

    \vfill
	\includegraphics[width=0.3\textwidth]{monster.png}\\[1cm] % Include a department/university logo.


    % Date

    \vfill % Position the date 3/4 down the remaining page

	{\large\today} % Date, change the \today to a set date if you want to be precise


    \vfill % Push the date up 1/4 of the remaining page

\end{titlepage}


% CONTENT - CHAPTERS


\pagenumbering{roman}


\pagestyle{fancy}


\renewcommand{\sectionmark}[1]{\markright{\thesection.\ #1}} %header conf. to show section

\renewcommand{\footrulewidth}{0pt}
\renewcommand{\headrulewidth}{0.4pt}



\fancyhead{}

\fancyfoot{}
\fancyfoot[RE]{\thepage}
\fancyfoot[LO]{\thepage}

\renewcommand{\contentsname}{Contenido}

\doublespacing
    \tableofcontents % Prints the main table of contents
        \addtocontents{toc}{~\hfill\textbf{\small{Página}}\par} % add pages to toc
\singlespacing


% PRE-SECTIONS


\newpage

% In case you need to add more lists, do it here

\renewcommand{\listfigurename}{Lista de figuras}
\renewcommand{\listtablename}{Lista de cuadros}

\listoffigures % Prints the list of figures
    \addcontentsline{toc}{section}{\listfigurename}

\listoftables % Prints the list of tables
    \addcontentsline{toc}{section}{\listtablename}

\newpage

\section*{Introducción}
    \addcontentsline{toc}{section}{Introducción}

\begin{figure}[htbp]
    \begin{center}
        \includegraphics[scale=0.5]{vim.png}
    \end{center}
\caption{vim logo}
\end{figure}



% BEGIN SECTIONS AREA


\newpage

\pagenumbering{arabic}

\fancyhead[RE,LO]{\textsl{\rightmark}} % Show the section/subsection in header

% Comment next line in case you need to reset equation counter in each section
%\counterwithin*{equation}{section}

\section{Example}

Here is an example of a table templated with calc2latex macro in LibreOffice.

\begin{table}[htbp]
\centering
    \begin{tabular}{cccc}
    \hline
    C $(mol\cdot cm^{-3})$ & $k\;(S/cm)$ & $\Lambda_{m}\;(S\cdot cm^{2}/mol)$ & $\sqrt{C} \; (mol^{1/2}\cdot cm^{-3/2})$ \\
    \hline
    \hline
    \cancel{$\mathbf{1.60\cdot 10^{-4}}$} & \cancel{$\mathbf{1.91\cdot 10^{-2}}$} & \cancel{$\mathbf{119.677}$} & \cancel{$\mathbf{1.26\cdot 10^{-2}}$} \\
    \cancel{$\mathbf{8.00\cdot 10^{-5}}$} & \cancel{$\mathbf{1.04\cdot 10^{-2}}$} & \cancel{$\mathbf{130.229}$} & \cancel{$\mathbf{8.94\cdot 10^{-3}}$} \\
    $4.00\cdot 10^{-5}$ & $5.23\cdot 10^{-3}$ & $130.706$ & $6.32\cdot 10^{-3}$ \\
    $2.00\cdot 10^{-5}$ & $2.79\cdot 10^{-3}$ & $139.413$ & $4.47\cdot 10^{-3}$ \\
    $1.00\cdot 10^{-5}$ & $1.50\cdot 10^{-3}$ & $149.706$ & $3.16\cdot 10^{-3}$ \\
    $5.00\cdot 10^{-6}$ & $7.93\cdot 10^{-4}$ & $158.670$ & $2.24\cdot 10^{-3}$ \\
    $2.50\cdot 10^{-6}$ & $4.03\cdot 10^{-4}$ & $161.276$ & $1.58\cdot 10^{-3}$ \\
    \hline
    \end{tabular}
    \caption{Magnitudes a baja concentración}
\end{table}


Some use of science packages:\\
    \newcommand{\laplace}{\mathcal{L}}
\[
\underbrace{\begin{bmatrix}
s+10 & -4\\
-4 & s+4\\
\end{bmatrix}}_{S}
\begin{bmatrix}
\laplace (x(t))\\
\laplace (y(t))\\
\end{bmatrix}
=
\begin{bmatrix}
1\\
-1\\
\end{bmatrix}
\]\\

\begin{center}
\ce{Na2SO3} + 2\ce{CuCl2} + \ce{H2O} $\longrightarrow$ \ce{Na2SO4} + 2\ce{CuCl} $\downarrow$ + 2\ce{HCl}
\end{center}


\begin{equation}
\Lambda_{m} \cdot C = \kappa = K_{c}\left( \frac{\Lambda_{m}^{0^2}}{\Lambda_{m}}-\Lambda_{m}^{0}\right)
\end{equation}

\begin{equation}
\sigma_{d}=\sqrt{\sum \limits_{j=1}^n \left(\frac{\partial d}{\partial x_j}\right)^{2}\,\sigma^{2}_{x_j}}\\[0.5cm]
\end{equation}

\setbohr{distribution-method=quantum,insert-missing}
\elconf{Fe}\\
\bohr{}{Fe}
\setbohr{distribution-method=periodic}
\bohr{}{Fe}



\newpage
\section{Example references}
Citing a website:

\begin{center}
    This is the reference that will appear in the webgraphy: \cite{windows}
\end{center}

Citing Einstein's article in the .bib file:

\begin{center}
    This is the reference: \cite{einstein}
\end{center}

You can make a footnote linked to the .bib file:
\begin{center}
    Redhat \autocite{redhat}
\end{center}
Or just a plain footnote:
\begin{center}
    Redhat\footnote{This one is not referenced to anything}.
\end{center}

\newpage


%   REFERENCES

%Add heading=bibintoc or subbibintoc to add in table of contents.

%\printbibliography[type=article,title={Articles only}]
%\printbibliography[type=book,title={Books only}]
%\printbibliography[keyword={physics},title={Physics-related only}]

\printbibliography[heading=bibintoc,title={Referencias}]
\printbibliography[heading=subbibintoc,type=online,title={Webgrafía}]
\printbibliography[heading=subbibintoc,type=article,title={Artículos}]


\end{document}


